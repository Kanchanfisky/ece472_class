
\documentclass[letterpaper,10pt,onecolumn,titlepage]{article}

\usepackage{graphicx}                                        
\usepackage{amssymb}                                         
\usepackage{amsmath}                                         
\usepackage{amsthm}                                          

\usepackage{alltt}                                           
\usepackage{float}
\usepackage{color}
\usepackage{url}

%\usepackage{balance}
%\usepackage[TABBOTCAP, tight]{subfigure}
%\usepackage{enumitem}
\usepackage{pstricks, pst-node, pst-tree}


\usepackage{geometry}
\geometry{textheight=8.5in, textwidth=6in}

% random comment

\newcommand{\cred}[1]{{\color{red}#1}}
\newcommand{\cblue}[1]{{\color{blue}#1}}

\usepackage{hyperref}
\usepackage{geometry}

% please replace your name here...
\def\name{Jacob Branaugh}

\parindent = 0.0 in
\parskip = 0.2 in

%% The following metadata will show up in the PDF properties
\hypersetup{
  colorlinks = false,
  urlcolor = black,
  pdfauthor = {Jacob Branaugh},
  pdfkeywords = {cs472 ``computer architecture''},
  pdftitle = {CS 472 Final},
  pdfsubject = {CS 472 Final},
  pdfpagemode = UseNone
}

\pagestyle{empty}
\begin{document}

\noindent {\large \bf Name: Jacob Branaugh \hfill ECE 472 Final}

\noindent {\large \bf ID\#: 931-724-634}

%%%% INTRO %%%%
\par
In today's age, computers are ubiquitous. Most homes in the U.S. have at least one PC or
laptop, single board computers are available for low prices, and microcontrollers can be
found almost anywhere. But not all electronic devices are equal. Many different processor
architectures exist and vary in different ways. In this course, we have been looking at
and discussing the ARM, or Advanced RISC Machine, architecture. ARM is one of the most
widely used architectures on the planet, comfortably dominating the mobile market
\cite{Bent}. In an attempt to gain a greater understanding of the ARM architecture, it 
will be compared and contrasted against the PowerPC architecture.

%%%% ARCHITECTURE HISTORY %%%%
\par
ARM, originally known as the Acorn RISC Machine, was first developed by Acorn Computers 
in 1985 \cite{Levy}. While other companies such as Intel and AMD were developing larger scale
chips, ARM was developing small scale simply due to start up budget issues. This early
division from competing companies landed arm in its dominant position in the smaller scale
market that it still holds today. ARM hit its stride in the early 90's, when it was
approached by both TI and Samsung. As of 2013, ARM has reported production of 37 billion
processors. ARM currently features the Cortex 32-bit architecture family, as well as
the 64-bit ARMv8-A architecture as of 2011 \cite{Grisenthwaite}. PowerPC, or Performance Optimization With
Enhanced RISC-Performance Computing, was initially created in 1991 by Apple, IBM, and
Motorola jointly. PowerPC is a RISC, superscalar architecture that was intended to be both
low cost and high performance \cite{Tech}. The goal of PowerPC in the consumer market was an
attempt to counter the popularity of Intel's CISC 80386, 80486, and upcoming Pentium
processors. PowerPC had success in Apple machines, as well as in video game consoles of
the 00's. Today, PowerPC exists as the Power ISA.

%%%% INSTRUCTION SET DESIGN %%%%
\par
Both ARM and PowerPC are RISC architectures. RISC stands for Reduced Instruction Set
Computing, and was designed with the idea in mind that simpler instruction are capable
of providing higher performance through faster execution of instructions. ARM has many
different ways it can address operations. ARM can address direct to memory, direct to a
register, to an immediate, indirect to a register with optional offset and incrementing,
and relative to the program counter \cite{Regina}. ARM address length in most flavors of the
architecture are 32 bits, while ARMv8-A is 64 bit. PowerPC, on the other hand, can only address
relative to the program counter, relative to a base register, and relative to a base
register with an additional register as an index \cite{Physinfo}. PowerPC supports both 32 and 64 bit
addresses. Both ARM and PowerPC can address using pages as small as 4KB.

%%%% DATAPATH DESIGN %%%%
\par
The ARM architecture has 16 general purpose registers. However, not all are usable as
general purpose registers. R13 contains the stack pointer, R14 contains the link register,
and R15 contains the program counter. Outside of these three, registers R0 through R12 are
available for use in a program. The PowerPC architecture has a total of 64 available
registers, 32 of which being general purpose registers and 32 being floating point
registers. ARM features three different eight stage pipelines. All pipelines have fetch 1, 
fetch 2, decode, and register read stages. Following these four, each has a different set 
of execution cycles. The first pipeline handles register to register operation, the second
handles multiplication and accumulation, and the third handles loading and storing. Most 
ARM instruction only require one cycle to execute. PowerPC has a 6 stage 
pipeline, but some instruction may take up to 3 execute cycles \cite{Abramson}. When a pipeline is 
full, an instruction is being completed every cycle, so the number of stages is irrelevant.
However, since ARM has fewer stages than PowerPC, it will not take as long to refill the
pipeline after flush. ARM does not use microcode, which allows it to run much more
efficiently. PowerPC, on the other hand, does use microcode. This adds an additional level
of power to the architecture, at the price of microcode being costly to execute.

%%%% MEMORY SUBSYSTEM %%%%
\par
As previously discussed, both ARM and PowerPC have internal registers. ARM processors have
16 general purpose registers, of which 13 are usable due to the stack pointer, link
pointer, and program counter. PowerPC has a total of 64 registers, half being general
purpose and half being floating point dedicated. While the first ARM processors did not
have caches, modern ARM processors can have up to 64K L1 cache and up to 4M L2 cache
depending on the processor core. The first PowerPC processor had a 32K L1 cache and
support for an external L2 cache. Modern POWER processors used in servers and
supercomputers can have up to 64K split L1 cache, 256K per core L2 cache, and 80M L3
cache \cite{IBM}. ARM uses a two level page table for 4KB pages in 1MB sections, while PowerPC 
uses parallel lookup through 4 or 8 sets. This allows for much faster lookups, as a lookup
hit in any of the sets will cause the search to end.

%%%% OTHER INTERESTING TIDBITS %%%%
\par
ARM has a few interesting features that are uncommon in other RISC architectures. First is
the ability to perform shifts/rotates and arithmetic/logic/moving instructions as a single
instruction. For example, performing the action of $a = a - 8 \times b$ would normally
require multiplying b by 8, and then subtracting from a. In ARM, a single instruction can
perform this, in that the normal subtraction instruction has a logical shift left by 3
instruction embedded within. Additionally, ARM's ability to use PC relative,
pre-incrementing, and post-incrementing address modes is not seen elsewhere as often. The
advantage of PC relative addressing is that the program doesn't have to be anchored
anywhere specifically in memory. Without PC relative addressing, a jump must be performed
by finding the address that the desired instruction corresponds to. With PC relative
addressing, the address relative to the current location is all that is needed. So instead
of jumping to 0x0034FA24 (for example), the jump is to current + 16 (for example). While
PowerPC does not support these features, it does support a few unique features of its own.
PowerPC was one of the earlier 64 bit RISC architectures available outside of the
supercomputing world \cite{Bishop}. This allowed for more general purpose registers than
competitors, as well as the possibility of more effective memory mapping. Another
advantage of the PowerPC architecture is the fact that it is a superscalar architecture.
What this means is that the processor is capable of executing two instructions
simultaneously. In a pipelined scenario (assuming the pipeline is full), this means that
instead of a single instruction being executed every cycle, there are two instructions
being executed every cycle. This allows for, theoretically, a doubling of performance.

%%%% HOW THE SYSTEM'S FEATURES CAN BOOST PERFORMANCE %%%%
\par
Part of the reason as to why ARM achieves the level of performance it does is the lack of
microcode. Though microcode can be a powerful tool, it is costly (in terms of resources)
to implement. Due to its absence, ARM can achieve much lower power draws, while keeping
performance up by focusing on caching for performance boosts. Additionally, ARM gets a
performance boost due to the fact that is has different pipelines for different types of
instructions. PowerPC gets a performance boost largely in part from the fact that it is a
superscalar architecture. The fact that instructions can be executed in direct parallel
means an increase in performance, as more tasks are being performed per given time.
However, this also means that even more time is wasted on a pipeline flush, as it is
dumping and then refilling twice the amount of data. It also leaves the processor much 
more open to pipeline hazards. An additional performance boost is gained from the way in
which memory is mapped. Since PowerPC implements parallel lookup, the average time taken
to perform a lookup is lower due to the fact that a hit in any set causes successful
lookup. This is especially important in high performance computing, where lookups are
constantly being performed.

\par
Though ARM is a powerful architecture, its power lies in the fact that it performs well
considering how low power it is. In terms of raw computing power, the PowerPC architecture
is the clear winner. Neither architecture is better than the other, and neither
architecture is really better or worse than any competing architecture. The effectiveness
of an architecture ultimately boils down to the application. In a low power application,
ARM may not even be as effective as some other, lower power architecture. Likewise,
PowerPC may not be the most effective architecture for a high performance computing
application. Understanding the pros and cons of various architectures is the key to
selecting the one that fits the problem best and to creating the best product.


\pagebreak
\begin{thebibliography}{9}
	\bibitem{Bent} Bent, Kristin. {\em ARM Snags 95 Percent Of Smartphone Market, Eyes 
			New Areas For Growth.} CRN. N.p., 165 July 2012. Web. 09 Dec. 2013.
			http://www.crn.com/news/components-peripherals/240003811/arm-snags-95-percent-of-smartphone-market-eyes-new-areas-for-growth.htm.
	\bibitem{Levy} Levy, Markus. {\em The History of The ARM Architecture: From Inception
			to IPO.} Convergence Promotions, n.d. Web. 9 Dec. 2013.
			http://www.reds.ch/share/cours/ReCo/documents/TheHistoryOfTheArmArchitecture.pdf.
	\bibitem{Grisenthwaite} Grisenthwaite, Richard (2011). {\em ARMv8-A Technology 
			Preview.} 9 Dec. 2013.
	\bibitem{Tech} {\em Tech Files Columns, 1987-1900.} 9 Dec. 2013.
	\bibitem{Regina} Regina, University of. {\em ARM Addressing Modes.} Addressing Lab.
			Department of Computer Science, n.d. Web. 09 Dec. 2013.
			http://www.cs.uregina.ca/Links/class-info/301/ARM-addressing/lecture.html.
	\bibitem{Physinfo} Physinfo. {\em Addressing Modes.} Addressing Modes. N.p., n.d. 
			Web. 09 Dec. 2013.
			http://physinfo.ulb.ac.be/divers\_html/powerpc\_programming\_info/intro\_to\_ppc/ppc3\_software2.html.
	\bibitem{Abramson} Abramson, David. {\em Power PC Pipeline Stages.} Power PC Pipeline
			Stages. University of Montana, 2000. Web. 09 Dec. 2013.
			http://www.csse.monash.edu.au/~davida/teaching/cse3304/Web/Chapter5/sld036.htm.
	\bibitem{IBM} {\em IBM POWER Microprocessors.} Wikipedia. Wikimedia Foundation, 12
			June 2013. Web. 09 Dec. 2013.
			http://en.wikipedia.org/wiki/IBM\_POWER\_microprocessors.
	\bibitem{Bishop} J. W. Bishop, et al.: {\em PowerPC AS A10 64-bit RISC 
			microprocessor}, IBM Journal of Research and Development, Volume 40, 
			Number 4, July 1996, IBM Corporation.
\end{thebibliography}

\end{document}
