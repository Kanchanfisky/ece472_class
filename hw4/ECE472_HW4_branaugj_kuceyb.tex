\documentclass[letterpaper,10pt,titlepage]{article}

\usepackage{graphicx}                                        
\usepackage{amssymb}                                         
\usepackage{amsmath}                                         
\usepackage{amsthm}                                          

\usepackage{alltt}                                           
\usepackage{float}
\usepackage{color}
\usepackage{url}

%\usepackage{balance}
%\usepackage[TABBOTCAP, tight]{subfigure}
%\usepackage{enumitem}
\usepackage{pstricks, pst-node}

\usepackage{geometry}
\geometry{textheight=9in, textwidth=6.5in}

%random comment

\newcommand{\cred}[1]{{\color{red}#1}}
\newcommand{\cblue}[1]{{\color{blue}#1}}

\usepackage{hyperref}
\usepackage{geometry}

\def\name{Jacob Branaugh, Brenn Kucey}

%% The following metadata will show up in the PDF properties
\hypersetup{
  colorlinks = true,
  urlcolor = black,
  pdfauthor = {\name},
  pdfkeywords = {cs472 ``computer architecture'' clements ``chapter 4''},
  pdftitle = {CS 472: Homework 4},
  pdfsubject = {CS 472: Homework 4},
  pdfpagemode = UseNone
}

\begin{document}
\hfill \name

\hfill \today

\hfill CS 472 HW 4

\begin{enumerate}
	\item[(9.2)] Why do computers use cache memory?
	\item[\textbullet] Computers use cache memory because it transfers data quickly,
		and has a much larger capacity than the processor's registers. Cache
		memory is slightly slower than registers, but is significantly faster than
		RAM. There are only a handful of registers available, whereas cache can
		range from kB to MB. RAM, however, is usually available in the order of
		GB.

	\item[(9.3)] What is the meaning of the following terms?
	\item[\textbullet] Temporal locality
	\item[-] Temporal locality describes the number of times a memory address is
		accessed within some amount of time.
	\item[\textbullet] Spatial locality
	\item[-] Spatial locality is a term to describe the physical closeness of related
		memory addresses.

	\item[(9.4)] From first principles, derive an expression for the speedup ratio of
		a memory system with cache (assume the hit ratio is h and the ratio of the
		main storage access time to cache access time is k, where k \textless\ 1). 
		Assume that the system is an ideal system and that you don't have to worry 
		about the effect of clock cycle times.
	\item[\textbullet] $S=\frac{1}{1-h(1-k)}\ ,\ k = \frac{t_{c}}{t_{m}}$

	\item[(9.5)] For the following ideal systems, calculate the speedup ratio S. In
		each case, $t_{c}$ is the access time of the cache memory, $t_{m}$ is the
		access time of the main store, and h is the hit ratio.
	\item[a)] $t_{m} = 70ns, t_{c} = 7ns, h = .9$ \\ \\
		$S=\frac{1}{1-.9(1-\frac{7}{70})}=$ \textbf{5.263}
	\item[b)] $t_{m} = 60ns, t_{c} = 3ns, h = .9$ \\ \\
		$S=\frac{1}{1-.9(1-\frac{3}{60})}=$ \textbf{6.897}
	\item[c)] $t_{m} = 60ns, t_{c} = 3ns, h = .8$ \\ \\
		$S=\frac{1}{1-.8(1-\frac{3}{60})}=$ \textbf{4.167}
	\item[d)] $t_{m} = 60ns, t_{c} = 3ns, h = .97$ \\ \\
		$S=\frac{1}{1-.97(1-\frac{3}{60})}=$ \textbf{12.739}

	\item[(9.6)] For the following ideal systems, calculate the hit ratio h required to
		achieve the stated speedup ratio S.
	\item[a)] $t_{m} = 60ns, t_{c} = 3ns, S = 1.1$ \\ \\
		$h=\frac{1-\frac{1}{S}}{1-\frac{t_{c}}{t_{m}}}=\frac{1-\frac{1}{1.1}}{1-\frac{3}{30}}=$
		\textbf{0.096}
	\item[b)] $t_{m} = 60ns, t_{c} = 3ns, S = 2.0$ \\ \\
		$h=\frac{1-\frac{1}{2}}{1-\frac{3}{60}}=$ \textbf{0.526}
	\item[c)] $t_{m} = 60ns, t_{c} = 3ns, S = 5.0$ \\ \\
		$h=\frac{1-\frac{1}{5}}{1-\frac{3}{60}}=$ \textbf{0.842}
	\item[d)] $t_{m} = 60ns, t_{c} = 3ns, S = 15.0$ \\ \\
		$h=\frac{1-\frac{1}{15}}{1-\frac{3}{60}}=$ \textbf{0.982}

	\item[(9.8)] For the following systems that use a clocked microprocessor,
		calculate the maximum speedup ratio you could expect to see as h approaches
		100\%.\\ \\
		$S=\frac{1}{1-1(1-\frac{t_{c}}{t_{m}})}=\frac{1}{\frac{t_{c}}{t_{m}}}=\frac{t_{m}}{t_{c}}$
		, where $t_{m}\ and\ t_{c}$ must be integer multiples of the time taken for
		one clock cycle.
	\item[a)] $t_{cyc} = 20ns, t_{m} = 75ns, t_{c} = 15ns$ \\ \\
		$t_{m}=75ns \rightarrow t_{m}=4\ cycles$ \\
		$t_{c}=15ns \rightarrow t_{c}=1\ cycle$ \\
		$S=\frac{4}{1}=$ \textbf{4}

	\item[b)] $t_{cyc} = 20ns, t_{m} = 75ns, t_{c} = 25ns$ \\ \\
		$t_{m}=75ns \rightarrow t_{m}=4\ cycles$ \\
		$t_{c}=25ns \rightarrow t_{c}=2\ cycle$ \\
		$S=\frac{4}{2}=$ \textbf{2}

	\item[c)] $t_{cyc} = 10ns, t_{m} = 75ns, t_{c} = 15ns$ \\ \\
		$t_{m}=75ns \rightarrow t_{m}=8\ cycles$ \\
		$t_{c}=15ns \rightarrow t_{c}=2\ cycle$ \\
		$S=\frac{8}{2}=$ \textbf{4}

	\item[(9.11)] In a direct-mapped cache memory system, what is the meaning of the
		following terms.
	\item[\textbullet] Word: the number of bytes a processor core can process during
		one clock cycle, usually no larger than 64 bits
	\item[\textbullet] Line: the minimum amount of data transferable between cache
		and main memory at a time, equal to a small number of words
	\item[\textbullet] Set: a collection of lines equal to $\frac{size\ of\ main\
		memory}{size\ of\ cache}$

	\item[(9.12)] What is the binary encoding of the following instructions?
		\begin{enumerate}
			\item[a)] STRB\ \ \ r1, [r2]
			\item[-] 1110 01 0 0 0 1 0 0 0010 0001 000000000000
			\item[b)] LDR\ \ \ r3, [r4, r5]!
			\item[-] 1110 01 1 1 1 0 1 1 0100 0011 00000 00 0 0101
			\item[c)] LDR\ \ \ r3, [r4], r5
			\item[-] 1110 01 1 0 1 0 1 1 0100 0011 00000 00 0 0101
			\item[d)] LDR\ \ \ r3, [r4, \#-6]!
			\item[-] 1110 01 0 1 0 0 1 1 0100 0011 111111111010
		\end{enumerate}

	\item[(9.17)] Write an ARM assembly language program that scans a string
		terminated by the null byte 0x00 and copies the string from a source
		location pointed at by r0 to a destination pointed at by r1.
		\\
		\begin{verbatim}
		START:                  ; start of loop
		LDR   r2, [r0], #4      ; load byte into r2, increment r0 pointer
		CMP   r2, #0x00         ; check if byte is null
		BEQ   END               ; jump to end of loop if null byte
		STRB  r2, [r1], #4      ; store read byte into r1, increment r1 pointer
		B     START             ; jump to start of loop
		END:                    ; end of loop, continue with program
		\end{verbatim}

	\item[(9.22)] Write an ARM assembly language program to determine whether a string
		of characters with and odd length is a palindrome under the following
		constraints: The string of ASIC-encoded characters is stored in memory, At
		the start of the program, register r1 contains the address of the first
		character in the string, and r2 contains the address of the last
		character. On exit from the program, register r0 contains a 0 if the
		string is not a palindrome, and 1 if it is.
		\\
		\begin{verbatim}
		START:                  ; start of loop
		CMP    r1, r2           ; check if registers are pointing to same location
		MOVEQ  r0, #1           ; load immediate 1 into r0 if same location
		BEQ    END              ; jump to end of loop
		LDR    r3, [r1], #4     ; load data at location in r1 to r3, increment r1
		LDR    r4, [r2], #-4    ; load data at location in r2 to r4, decrement r2
		CMP    r3, r4           ; compare data at those locations
		MOVNE  r0, #0           ; store immediate 0 to r0 if not same character
		BNE    END              ; jump to end of loop
		B      START            ; jump to start of loop
		END:                    ; end of loop, continue with program
		\end{verbatim}
	\item[(9.23)]
	\item[(9.26)]
	\item[(9.28)]
	\item[(9.35)]
	\item[(9.41)]
	\item[(9.42)]
	\item[(9.43)]
	\item[(9.45)]
	\item[(9.46)]
	\item[(9.57)]

\end{enumerate}

\end{document}
