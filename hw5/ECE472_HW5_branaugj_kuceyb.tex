\documentclass[letterpaper,10pt,titlepage]{article}

\usepackage{graphicx}                                        
\usepackage{amssymb}                                         
\usepackage{amsmath}                                         
\usepackage{amsthm}                                          

\usepackage{alltt}                                           
\usepackage{float}
\usepackage{color}
\usepackage{url}

%\usepackage{balance}
%\usepackage[TABBOTCAP, tight]{subfigure}
%\usepackage{enumitem}
\usepackage{pstricks, pst-node}

\usepackage{geometry}
\geometry{textheight=9in, textwidth=6.5in}

%random comment

\newcommand{\cred}[1]{{\color{red}#1}}
\newcommand{\cblue}[1]{{\color{blue}#1}}

\usepackage{hyperref}
\usepackage{geometry}

\def\name{Jacob Branaugh, Brenn Kucey}

%% The following metadata will show up in the PDF properties
\hypersetup{
  colorlinks = true,
  urlcolor = black,
  pdfauthor = {\name},
  pdfkeywords = {cs472 ``computer architecture'' clements ``chapter 5''},
  pdftitle = {CS 472: Homework 5},
  pdfsubject = {CS 472: Homework 5},
  pdfpagemode = UseNone
}

\begin{document}
\hfill \name

\hfill \today

\hfill CS 472 HW 5

\begin{enumerate}
	\item[(6.5)] The time taken by machines A, B, and C to execute a given task is
	\item[-] A\ \ 16m, 9s
	\item[-] B\ \ 14m, 12s
	\item[-] C\ \ 12m, 17s
		What is the performance of each of these machines relative to machine A?
	\item[\textbullet] $Perf_{AtoA} = \frac{969s}{969s} = 1$
	\item[\textbullet] $Perf_{BtoA} = \frac{969s}{852s} = 1.14$, 14\% faster
	\item[\textbullet] $Perf_{CtoA} = \frac{969s}{737s} = 1.31$, 31\% faster

	\item[(6.6)] Why is clock rate a poor metric of computer performance? What are the
		relative strengths and weaknesses of clock speed as a performance metric?
	\item[\textbullet] All clock speed tells you is how often the processor clock
		ticks. If two processors are identical, but one has a slightly higher
		clock speed, it is usually an indicator that the faster processor will
		perform slightly better. Factors such as cache sizes, architecture,
		computing power of the processor core, and power draw have a much greater
		say in determining how well a processor performs, as these listed
		attributes have a much greater effect on performance.

	\item[(6.12)] The following figures define the typical operating parameters of a
		processor:\\
		\\
		\begin{tabular}{ l r c }
			Operation & Frequency & Cycles \\
			\hline
			Arithmetic/logic instructions & 45\% & 1 \\
			     Register load operations & 20\% & 3 \\
			    Register store operations & 10\% & 2 \\
			      All branch instructions & 25\% & 2 \\
		\end{tabular} \\ \\
		If the clock rate could be reduced by 15\%, it would require only 2 cycles
		to perform a register load. Would that be a good idea?
	\item[\textbullet]  Yes. While a drop in clock speed will only make a small
		performance distance at best, wasting one less cycle on load operations
		will boost the overall system performance. 

	\item[(6.13)] A computer has the following parameters:\\
		\\
		\begin{tabular}{ l r c }
			Operation & Frequency & Cycles \\
			\hline
			Arithmetic/logic instructions & 65\% & 1 \\
			     Register load operations & 10\% & 5 \\
			    Register store operations &  5\% & 2 \\
			      All branch instructions & 20\% & 8 \\
		\end{tabular} \\ \\
		If the average performance of the computer (in terms of its CPI) is to be
		increased by 20\% while executing the same instruction mix, what target
		must be achieved for the cycles per conditional branch instruction?
	\item[\textbullet] $CPI_{A} = .65 + .5 + .1 + 1.6 = 2.85$
	\item[\textbullet] $CPI_{B} = \frac{CPI_{A}}{1.2} = 2.375$
	\item[\textbullet] $cycles_{branch} \leq \frac{2.375-.65-.5-.1}{.2} = 5.625
		\Rightarrow cycles_{branch} = $ \textbf{5}

	\item[(6.17)] For the following systems that have both serial and parallel
		activities, calculate the speedup ratio.
	\item[-] 10 processors, $f_{s} = 0.1$
	\item[\textbullet] $S = \frac{p}{pf_{s}+1-f_{s}} = \frac{10}{10\times0.1+1-0.1} =
		5.26$
	\item[-] 100 processors, $f_{s} = 0.1$
	\item[\textbullet] $S = \frac{100}{100\times0.1+1-0.1} = 9.17$
	\item[-] 5 processors, $f_{s} = 0.4$
	\item[\textbullet] $S = \frac{5}{5\times0.4+1-0.4} = 1.92$
	\item[-] 100 processors, $f_{s} = 0.01$
	\item[\textbullet] $S = \frac{100}{100\times0.01+1-0.01} = 50.25$

	\item[(6.18)] A system has a single core processor that costs \$150. Suppose that
		adding more cores to the chip costs \$10 per additional processor.
		(\textit{Note}: For this system, the value of $f_{s}$ is 0.1). If it is
		considered worthwhile adding cores until the incremental speedup ratio
		increases by less than 5\% over the original performance, what is the
		optimum number of processors? What percentage increase in cost is required
		to achieve this performance?
	\item[\textbullet] $S_{i} = \frac{1}{1\times0.1+1-0.1} = 1$
	\item[\textbullet] Since the initial speed up is 1, we need to increase the number
		of processors until the speed up increase is less than .05. We wrote a
		quick python program to calculate this, and the event occurs at 34
		processors, which means we want a 33 core processor. The additional 32
		processors cost \$320, bringing the total cost to \$470 and the cost
		percentage increase to about 313\%.

	\item[(6.22)] A coprocessor is added to a computer to speed the execution time of
		string-processing instructions by a factor of 3.5. What fraction of the
		execution time must use these string-processing instructions in order to
		achieve an average speed up of 1.5?
	\item[\textbullet] $S = \frac{1}{1-fraction+\frac{fraction}{factor}} \Rightarrow
		fraction = \frac{\frac{1}{S}-1}{\frac{1}{factor}-1} = .466$
	\item[\textbullet] 46.6\% of the execution time must be used on string processing
		to achieve the desired speed up.

	\item[(7.12)] Figure 7.12 from the text demonstrates the execution of a
		conditional branch instruction in a flow-through computer. The grayed out
		sections of the computer are not required by a conditional branch
		instruction. Can you think of any way in which these unused elements of the
		computer could be used during the execution of a conditional branch?
	\item[\textbullet] They could be used to perform arithmetic operations on register
		contents and store them back in the register file. A memory store could
		also be performed.

	\item[(7.15)] What modification would have to be made to the architecture of the
		computer in the figure to implement operand shifting (as part of a normal
		instruction) like the ARM?
	\item[\textbullet] Modifications have to be made to allow a literal to be passed
		in with a register. Multiplying or dividing by a literal power of 2 will
		shift left or right, respectively.

\end{enumerate}

\end{document}
